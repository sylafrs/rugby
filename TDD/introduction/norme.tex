\section{Norme}
Pour s'y retrouver plus facilement, nous avons établi une norme basée sur les contraintes d'Unity et de ce document : \href{http://devmag.org.za/2012/07/12/50-tips-for-working-with-unity-best-practices/}{50 tips for working with unity}

\subsection{Scripts}

Ils seront en priorité écrits avec C\#.
On utilisera les autres langages que si nous n'avons pas le choix.

\subsection{Hiérarchie}
\subsubsection{Fichiers}

Tout type de fichier doivent se retrouver ensemble.
Les textures avec les textures, les scripts avec les scripts, etc..

Il ne faudra pas non plus hésiter à créer des dossiers, peu importe si le chemin pour accéder aux fichiers est long.

\subsubsection{Scène}

Une fois de plus : ne pas hésiter à ranger les GameObjects dans des GameObjects vides.
Les scripts demanderont quel GameObject devra accueillir quels GameObjects dynamiques

Tout, sauf les GameObjects vides, devront être instanciés par des prefab.

\subsection{Nommage}
Tout ce qui peut, dans le projet, suivre cette convention, devra suivre cette convention de nommage :

\begin{enumerate}
\item Tout devra être en anglais.
\item Tout devra commencer par une majuscule, ne contenir aucun caractere special (comme l'underscore "\_" ou l'espace).
\item Si le nom contient plusieurs mots, les coller ensemble et le faire commencer par une majuscule.
\item On part de l'élément le plus spécifique
\item Aucune abréviation.
\item Le fichier lié à l'objet devra avoir le même nom de base
\end{enumerate}

Exemple :
\begin{itemize}
\item RedButton   : OK
\item ButtonRed   : Red est l'élément le plus spécifique
\item RedBtn      : Pas d'abréviation
\item redButton   : On commence par une majuscule
\item Red Button  : Pas d'espace
\item Red\_Button  : Pas d'underscore ou autre caractère spécial
\item Redbutton   : Faire commencer le second mot button par une majuscule 
\item RougeBouton : En anglais
\end{itemize}

\subsubsection{Version}

Dans le cas où un fichier contient plusieurs versions :
\begin{enumerate}
	\item Les fichiers archivés devront se situer dans le dossier Version, puis dans un dossier ayant le nom de base
	\item Les fichiers utilisés seront dans le dossier /Archive
	\item Les fichiers auront pour suffixe "\_X" où X est l'indice de version
	\item L'indice de version a pour seule contrainte de devoir suivre l'ordre alphabétique (table ascii) mais peut ne pas suivre les règles de convention de nommage citées ci-dessus
\end{enumerate}

Exemple :
\dirtree{%
 .1 /.
 .2 Archive/.
 .3 Prefabs/.
 .4 Buttons/.
 .5 RedButton/.
 .6 RedButton\_v1.0\_201210121234.
 .6 RedButton\_v1.0\_201210040930.
 .2 Current/.
 .3 Prefabs/.
 .4 Buttons/.
 .5 RedButton.
}
\newpage
\subsubsection{Extension de fichier}
Pour les fichiers, le nom de base devra respecter la convention de nommage ci-dessus et devra contenir une extension selon le type du fichier.\\

%
% +----------+-------------+-----------------------------------------+
% |   Type   |  Extension  |              Description                |
% +----------+-------------+-----------------------------------------+
%

\begin{tabular}{|p{2.5cm}|p{1cm}|p{7cm}|}
	\hline
		\rowcolor{table_header_color} 
		\begin{bf}\begin{center}Type\end{center}\end{bf} & \begin{bf}\begin{center}Ext.\end{center}\end{bf} & \begin{bf}\begin{center}Description\end{center}\end{bf} \\
	\hline
		\multirow{3}*{Script en C\#} & \multirow{3}*{.cs} & \multirow{3}*{Morceau de code utilisé par unity} \\
		& & \\
		& & \\
	\hline
		\multirow{3}*{Image} & .jpg & \multirow{3}*{Image utilisée pour une texture ou une map}\\
		 & .png & \\
		 & .psd & \\
	\hline
		\multirow{3}*{GameObjects} & \multirow{3}*{.prefab} & \multirow{3}*{Les GameObjects préenregistrés} \\
		& & \\
		& & \\
	\hline
		\multirow{3}*{Scene} & \multirow{3}*{.unity} & \multirow{3}*{Une scène (Ex : menu, terrain, ..)}\\		
		& & \\
		& & \\
	\hline
		\multirow{3}*{Shader} & \multirow{3}*{.shader} & \multirow{3}*{Méthode de rendu d'un material} \\	
		& & \\
		& & \\
	\hline
		\multirow{3}*{Material} & \multirow{3}*{.mat} & \multirow{3}*{Une matière (shader paramétré)} \\	
		& & \\
		& & \\			
	\hline
		\multirow{3}*{Animation} & \multirow{3}*{.fbx} & \multirow{3}*{Mouvement du RIG\footnotemark[1] d'un objet}\\
		& & \\
		& & \\
	\hline
\end{tabular}

\footnotetext[1]{Squelette d'un objet, lié au meshes pour permettre une animation}